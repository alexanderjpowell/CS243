\documentclass[11pt]{article}

\usepackage{times,mathptm}
\usepackage{pifont}
\usepackage{exscale}
\usepackage{latexsym}
\usepackage{amsmath}
\usepackage{epsfig}
\usepackage{amsthm}

\textwidth 6.5in
\textheight 9in
\oddsidemargin -0.0in
\topmargin -0.0in

\parindent 0pt     % How much the first word of a paragraph is indented. 
\parskip 0pt	   % How much extra space to leave between paragraphs.

\begin{document}

\begin{center}             % If you only centering 1 line use \centerline{}
\begin{LARGE}
{\bf CSci 243 Homework 1}
\end{LARGE}
\vskip 0.25cm      % vertical skip (0.25 cm)

Due: 10:00 am, Wednesday, Sep 9 \\  % force new line
Alexander Powell
\end{center}


\begin{enumerate}

\item Construct a truth table for each of these compound propositions.
\begin{enumerate}
\item (Rosen 1.1/31(c), 3 points) $(p\lor \lnot q)\rightarrow q$

\begin{center}
\begin{tabular} {|c|c|c|c|c|}
\hline
$p$ & $q$ & $\lnot q$ & $p \lor \lnot q$ & $(p \lor \lnot q) \rightarrow q$ \\
\hline
$T$ & $T$ & $F$ & $T$ & $T$ \\
$T$ & $F$ & $T$ & $T$ & $F$ \\
$F$ & $T$ & $F$ & $F$ & $T$ \\
$F$ & $F$ & $T$ & $T$ & $F$ \\
\hline
\end{tabular}
\end{center}

\item (7 points) $(p\leftrightarrow q)\rightarrow (r\leftrightarrow s)$

\begin{center}
\begin{tabular} {|c|c|c|c|c|c|c|}
\hline
$p$ & $q$ & $r$ & $s$ & $p \leftrightarrow q$ & $r \leftrightarrow s$ & $(p \leftrightarrow q) \rightarrow (r \leftrightarrow s)$ \\
\hline
$T$ & $T$ & $T$ & $T$ & $T$ & $T$ & $T$\\
$T$ & $T$ & $T$ & $F$ & $T$ & $F$ & $F$\\
$T$ & $T$ & $F$ & $T$ & $T$ & $F$ & $F$\\
$T$ & $T$ & $F$ & $F$ & $T$ & $T$ & $T$\\
$T$ & $F$ & $T$ & $T$ & $F$ & $T$ & $T$\\
$T$ & $F$ & $T$ & $F$ & $F$ & $F$ & $T$\\
$T$ & $F$ & $F$ & $T$ & $F$ & $F$ & $T$\\
$T$ & $F$ & $F$ & $F$ & $F$ & $T$ & $T$\\
$F$ & $T$ & $T$ & $T$ & $F$ & $T$ & $T$\\
$F$ & $T$ & $T$ & $F$ & $F$ & $F$ & $T$\\
$F$ & $T$ & $F$ & $T$ & $F$ & $F$ & $T$\\
$F$ & $T$ & $F$ & $F$ & $F$ & $T$ & $T$\\
$F$ & $F$ & $T$ & $T$ & $T$ & $T$ & $T$\\
$F$ & $F$ & $T$ & $F$ & $T$ & $F$ & $F$\\
$F$ & $F$ & $F$ & $T$ & $T$ & $F$ & $F$\\
$F$ & $F$ & $F$ & $F$ & $T$ & $T$ & $T$\\
\hline
\end{tabular}
\end{center}

\end{enumerate}

\item Using logical identities and laws, show the logic equivalence of
\begin{enumerate}
\item (Rosen 1.3/16, 5 points) $p\leftrightarrow q$ 
			and $(p\land q)\lor(\lnot p\land \lnot q)$
            
\begin{proof}

$$ p\leftrightarrow q $$
$$ \equiv (p \rightarrow q) \land (q \rightarrow p) $$
$$ \equiv (\lnot q \lor p) \land (\lnot p \lor q) $$
$$ \equiv \lnot q \land (\lnot p \lor q) \lor p \land (\lnot p \lor q) $$
$$ \equiv (p \land q) \lor (\lnot p \land \lnot q) \lor (p \land \lnot p) \lor (q \land \lnot q) $$
$$ \equiv (p \land q) \lor (\lnot p \land \lnot q) $$

\end{proof}

\item (Rosen 1.3/28, 5 points) $p\leftrightarrow q$ and $\lnot p \leftrightarrow \lnot q$

\begin{proof}

$$ p\leftrightarrow q $$
$$ \equiv (q \rightarrow p) \land (p \rightarrow q) $$
$$ \equiv (p \lor \lnot q) \land (q \lor \lnot p) $$
$$ \equiv (\lnot(\lnot p) \lor \lnot q) \land (\lnot(\lnot q) \lor \lnot p) $$
$$ \equiv (\lnot p \rightarrow \lnot q) \land (\lnot q \rightarrow \lnot p) $$
$$ \equiv \lnot p \leftrightarrow \lnot q  $$

\end{proof}

\end{enumerate}

\item Understanding quantified predicates.
\begin{enumerate}
\item (3 points) English to quantified predicates: 
Use predicates, quantifiers, logical and mathematical operators to express statement:
``The difference of two negative integers is not necessarily negative''.

{\bf Solution:}

$\exists n < 0$, $\exists m < 0$ $(n - m \geq 0)$

\item (7 points) Quantified predicate to English:
Give the truth value of each of these statement if the domain of all variables consists of all real numbers. 
\begin{enumerate}
\item $\forall n\exists m (n^2 < m)$

For any number $n$, there exists some other number $m$ such that $m$ is greater than $n^2$.  

\item $\exists n\forall m (n<m^2)$

There exists some number $n$ such that for all $m$, $n$ is less than $m^2$.  

\item $\forall n\exists m (n+m = 0)$

For any number $n$ there exists some number $m$ such that the sum of $n$ and $m$ equals $0$.  

\item $\exists n\forall m (nm = m)$

There exists some number $n$ such that for all $m$ the product of $n$ and $m$ equals $m$.  

\item $\exists n\exists m (n^2+m^2 = 4)$

There exist two numbers, $n$ and $m$ such that the sum of their squares equals $4$.  

\item $\exists n\exists m (n+m = 4 \land n-m=1)$

There exist two numbers, $n$ and $m$ such that the sum of $n$ and $m$ equals $4$ and $n$ minus $m$ equals $1$.  

\item $\forall n\forall m \exists p(p = (m+n)/2)$

For any number $n$ and for any number $m$ there exists some other number $p$ such that $p$ equals the sum of $m$ and $n$ divided by $2$.  

\end{enumerate}
\end{enumerate}

\newpage

\item (Rosen 1.5/30, 10 points) Rewrite each of these statements so that negations appear only within predicates, i.e., so that no negation is outside a quantifier or an expression involving logical operators.
\begin{enumerate}
\item 
$$\lnot\forall x \forall y P(x,y)$$
$$ \equiv \exists x \exists y \lnot P(x,y) $$

\item 
$$\lnot\forall y \exists x P(x,y)$$
$$ \equiv \exists y \forall x \lnot P(x,y) $$

\item 
$$\lnot\forall y \forall x (P(x,y) \lor Q(x,y))$$
$$ \equiv \exists y \exists x \lnot (P(x,y) \lor Q(x,y)) $$
$$ \equiv \exists y \exists x (\lnot P(x,y) \land \lnot Q(x,y)) $$

\item 
$$\lnot(\exists x\exists y\lnot P(x,y) \land \forall x\forall y Q(x,y))$$
$$ \equiv \lnot (\exists x \exists y \lnot P(x,y)) \lor \lnot (\forall x \forall y Q(x,y))$$
$$ \equiv (\forall x \forall y \lnot (\lnot P(x,y))) \lor (\exists x \exists y \lnot Q(x,y)) $$
$$ \equiv (\forall x \forall y P(x,y)) \lor (\exists x \exists y \lnot Q(x,y)) $$

\item 
$$\lnot\forall x(\exists y \forall z P(x,y,z) 
                        \land \exists z\forall y P(x,y,z))$$
$$ \equiv \exists x \lnot (\exists y \forall z P(x,y,z) \land \exists z \forall y P(x,y,z)) $$
$$ \equiv \exists x \lnot (\exists y \forall z P(x,y,z)) \lor \lnot (\exists z \forall y P(x,y,z))$$
$$ \equiv \exists x ((\forall y \exists z \lnot P(x,y,z)) \lor (\forall z \exists y \lnot P(x,y,z)))$$
                        
\end{enumerate}

\end{enumerate}

\end{document}

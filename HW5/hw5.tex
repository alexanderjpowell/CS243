\documentclass[11pt]{article}

\usepackage{times,mathptm}
\usepackage{pifont}
\usepackage{exscale}
\usepackage{latexsym}
\usepackage{amsmath}
\usepackage{amsthm}
\usepackage{amssymb}
\usepackage{epsfig}

\textwidth 6.5in
\textheight 9in
\oddsidemargin -0.0in
\topmargin -0.0in

\parindent 0pt     % How much the first word of a paragraph is indented. 
\parskip 0pt       % How much extra space to leave between paragraphs.

\begin{document}

\begin{center}             % If you only centering 1 line use \centerline{}
\begin{LARGE}
{\bf CSci 243 Homework 5}
\end{LARGE}
\vskip 0.25cm      % vertical skip (0.25 cm)

Due: 10:00 am, Wednesday, Oct 7 \\  % force new line
Alexander Powell
\end{center}

\begin{enumerate}

\item (10 points) Arrange the functions 
$$3^n, 2^n, n 2^n,  n^{30}, (\log n)^3, \sqrt{n}\log^2 n, n\log n, \sqrt{n!}, n^{29}+n^{27}, n^{2\sqrt{n}}$$
into increasing order of growth rates.

Arranging the functions in increasing order of growth we get:

$$ (\log n)^3 < \sqrt{n}\log^2 n < n\log n < n^{29}+n^{27} < n^{30} < n^{2\sqrt{n}} < 2^n < n 2^n < 3^n < \sqrt{n!} $$

\item (10 points) 
To solve a particular problem you have access to two algorithms. 
The execution time of the first algorithm can be given as a function of the
input size $n$ as $f(n) = n^{1.5} \log^2 n$.
The execution time of the second algorithm is similarly: $g(n) = n^2$.
Which algorithm is faster asymptotically?
Is this algorithm faster for small $n$?
Find the minimum problem size $n$ needed so that the fastest asymptotic 
algorithm becomes faster than the other one.
Hint: limit your search in powers of 2. You may use calculators to help you
but the answer must be self contained.

By plugging in some small values for $n$ we see that initially $f(n)$ appears to grow faster.  However, since $g(x)$ has the highest power, it seems likely that for an big enought input, $g(x)$ will eventually overtake $f(n)$.  To see if this is true, we can look for an itersection of the two functions.  We do this by setting them equal and taking the logarithm of both sides:

$$ n^{1.5}\log^2 n = n^2 $$
$$ \log{n^{1.5}\log^2 n} = \log{n^2} $$
$$ 1.5 \log n + \log{(\log^2 n)} = 2 \log n $$
$$ \log{(\log^2 n)} = \frac{1}{2} \log n $$
$$ \log{(\log^2 n)} = \log{(\sqrt{n})} $$
$$ \log^2 n = \sqrt{n} $$
$$ \log n = n^{.5 \times .5} = n^{.25} $$
$$ 2^{n^{.25}} = n $$

Now, we have gotten our $n$ in a form of a power of $2$ so we can limit our search to that.  If we start our search with $n=2^{14}$ and go from there we get the following results.  

\begin{center}
\begin{tabular}{ |c|c|c| } 
\hline
$n$ & $f(n)$ & $g(n)$ \\
\hline
$2^{14} = 16384$  & $411041792$   & $268435456$   \\
$2^{15} = 32768$  & $1334619360$  & $1073741824$  \\
$2^{16} = 65536$  & $4294967296$  & $4294967296$  \\
$2^{17} = 131072$ & $13713955382$ & $17179869184$ \\
\hline
\end{tabular}
\end{center}

From these points, we can see that the two functions cross exactly when $n = 2^{16}$ or $65536$.  For points smaller than $65536$, $f(n)$ takes more time but for inputs larger than $65536$, $g(n)$ takes more time.  Therefore, we can say that $g(n)$ has a greater asymptotic complexity which means that $f(n)$ is asymptotically faster, even though it is not faster for small $n$, specifically those smaller than $65536$.  

Because the two algorithms have the same speed when operating on a problem size of $n = 2^{16}$ or $65536$, then the minimum problem size $n$ needed so that $f(n)$ becomes faster is one more than that, or $65537$.  

\item (10 points) 
What is the largest problem size $n$ that we can solve in no more than 
{\bf  one hour} using an algorithm that requires $f(n)$ operations, 
where each operation takes $10^{-9}$ seconds (this is close to a today's computer), 
with the following $f(n)$?

First, it will be easier to deal with everything in terms of seconds, so there are $3600$ seconds in one hour.  Now, we just need to multiply every $f(n)$ by $10^{-9}$, set it equal to $3600$ and solve for $n$.  

\begin{enumerate}
\item $\log_2 n$

\textbf{Solution: }

$$3600 = \log_2 n \times 10^{-9}$$

$$\frac{3600}{10^{-9}} = \log_2 n \longrightarrow n = 2^{\dfrac{3600}{10^{-9}}}$$

\item $\log^4_2 n$

\textbf{Solution: }

Similarly to the above problem, we get

$$n = 2^{\sqrt[4]{\frac{3600}{10^{-9}}}}$$

\item $3n$

\textbf{Solution: }

$$3600 = 3n \times 10^{-9} \longrightarrow n = \dfrac{3600}{3 \times 10^{-9}}$$

\item $n\log_2 n$

\textbf{Solution: }

$3600 = n\log_2 n \times 10^{-9}$

$\frac{3600}{10^{-9}} = \log_2 n$

$$ 2^{\frac{3600}{10^{-9}}} = n $$

From here it is very difficult to solve for $n$, however we can conclude that it is an extremely large number

\item $n\log^2_2 n$

\textbf{Solution: }

$3600 = n\log^2_2 n \times 10^{-9}$

Again, we can rewrite this equation to look like 
$$ 2^{\sqrt{\frac{3600}{10^{-9}}}} = n $$
Which can't easily be solved by hand but again, we get a large number, meaning this is a very good algorithm to use when working with large problem sizes.  

\item $n^2$

\textbf{Solution: }

$$3600 = n^2 \times 10^{-9} \longrightarrow n = \sqrt{\dfrac{3600}{10^{-9}}}$$

\item $(3n)^3$

\textbf{Solution: }

$$3600 = (3n)^3 \times 10^{-9} \longrightarrow n = \dfrac{\sqrt[3]{\dfrac{3600}{10^{-9}}}}{3}$$

\item $2^n$

\textbf{Solution: }

$3600 = 2^n \times 10^{-9}$

$\log_2{3600} = \log_2{(2^n \times 10^{-9})}$

$\log_2{3600} = \log_2{(2^n + \log_2 10^{-9})}$

$\log_2{3600} = n + \log_2{10^{-9}}$

$n = \log_2 3600 - \log_2{10^{-9}}$

$n = \log_2{\bigg(\dfrac{3600}{10^{-9}}\bigg)}$

\item $n!$

\textbf{Solution: }

Because $n!$ is so fast growing, it's not necessary to compute this algebraicly.  By inputting a few numbers into the equation:
$$ 3600 = n! \times 10^{-9} $$
we can quickly see what the largest problem size possible is before going over an hour.  An input of size $14$ will take $14! \times 10^{-9} \approx 87$ seconds.  Input size $15$ gives us $15! \times 10^{-9} \approx 1307$ seconds.  The next largest input size, $16$, gives us the following: $16! \times 10^{-9} \approx 20922$ seconds.  Clearly $20922$ is greater than $3600$ so the largest problem size $n$ that takes no longer than one hour is $n = 15$.  

\item $n^n$

\textbf{Solution: }

We will use a process similar to the previous problem to determine the max problem size.  We know that $n^n$ grows faster than $n!$ so we can start guessing at numbers a little lower than before.  Let's begin with a guess of $n=10$, then $10^{10} \times 10^{-9} = 10$.  When $n=11$ we have $11^{11} \times 10^{-9} \approx 285$.  When $n=12$ we have $12^{12} \times 10^{-9} \approx 8916$, which is clearly over $3600$, so the largest problem size $n$ that takes no longer than one hour is $n = 11$.  

\end{enumerate}

\item (10 points) Use pseudocode to describe an algorithm that determines whether a given function from a finite set to another finite set is one-to-one.

Hint: You may assume that the domain is $A=\{a_1,\ldots,a_m\}$ and the co-domain is $B=\{b_1,\ldots,b_n\}$. The function $f: A\rightarrow B$ is given as a set of pairs $\{(a_i,f(a_i))|\forall a_i\in A\}$.
\\

To determine if the given function is one-to-one, we just need to determine if there are two or more inputs that map to the same output.  In other words, we need to examine our input elements in $f(a_i)$ and determine if any duplicates exist.  If there are no duplicates in $f(a_i) \forall a_i \in A$, we can conclude the function is one-to-one.  If at least one duplicate does exist, the function is not one-to-one.  The pseudocode for the algorithm is shown below.  

\begin{verbatim}
Inputs: Two arrays, one of a_i and one of f(a_i), where the input and output 
mappings are related through their index position.  

var length = lengthOfArray(a_i)
for (int x = 0; x < length; x++)
{
    for (int y = x; y < length; y++)
    {
        if (f(a_i)[x] == f(a_i)[y])
        {
            return false;
        }
    }
}
return true;
\end{verbatim}

Also, it's important to note this algorithm assumes it's given a valid function as input and there are no redundant inputs in the set of pairs (meaning two identical ordered pairs are not listed twice).  

\end{enumerate}

\end{document}

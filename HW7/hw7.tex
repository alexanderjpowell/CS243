\documentclass[11pt]{article}

\usepackage{times,mathptm}
\usepackage{pifont}
\usepackage{exscale}
\usepackage{latexsym}
\usepackage{amsmath}
\usepackage{amsthm}
\usepackage{amssymb}
\usepackage{epsfig}

\textwidth 6.5in
\textheight 9in
\oddsidemargin -0.0in
\topmargin -0.0in

\parindent 0pt     % How much the first word of a paragraph is indented. 
\parskip 0pt       % How much extra space to leave between paragraphs.

\begin{document}

\begin{center}             % If you only centering 1 line use \centerline{}
\begin{LARGE}
{\bf CSci 243 Homework 7}
\end{LARGE}
\vskip 0.25cm      % vertical skip (0.25 cm)

Due: 10:00 am, Wednesday, October 21 \\  % force new line
Alexander Powell
\end{center}

\begin{enumerate}

\item For each of the following program fragments, give an analysis of 
the running time. You may use summations to evaluate the running times 
of nested loops. Use big-O notation (i.e., ignore constants). Show your work.

\begin{enumerate}

\item (5 points)
\begin{verbatim}
for i = 1 to n
   for j = 1 to n
      if (i*j >= n)
         for k = 1 to n
             sum++
      endif
\end{verbatim}

\textbf{Solution: }

The following program can be written in the pseudo-summation shown below:
$$ \sum_{i=1}^{n} \sum_{j=1}^{n} \bigg( \text{ if } i \times j \geq n: \sum_{k=1}^{n} 1 \bigg) $$
which is equivalent to taking the entire summation and then subtracting the times when $i \times j < n$.  This is shown below:
$$ \sum_{i=1}^{n} \sum_{j=1}^{n} \bigg( \text{ if } i \times j \geq n: \sum_{k=1}^{n} 1 \bigg) = \sum_{i=1}^{n} \sum_{j=1}^{n} \sum_{k=1}^{n} 1 - \sum_{i=1}^{n} \sum_{j=1}^{n} \bigg( \text{ if } i \times j < n: \sum_{k=1}^{n} 1 \bigg) $$
We can now rewrite this line as 
$$ n^3 - \sum_{i=1}^{n} \sum_{j=1}^{n} \bigg( \text{ if } i \times j < n: \sum_{k=1}^{n} 1 \bigg) $$
Because the second term in the previous line has a power of something less than three, it becomes insignificant as $n$ grows towards $\infty$.  Therefore, the $n^3$ term remains as the highest power so we can describe the run time complexity as $O(n^3)$.  

\newpage

\item (5 points)
\begin{verbatim}
sum = 0
for i = 0 to n-1
   for j = 0 to i 
      for k = 1 to n/(2^j)          (2^j) means 2 to the power j
         sum++
\end{verbatim}

\textbf{Solution: }

We can write the program in the form of the triple sum shown below:
$$ \sum_{i=0}^{n-1} \sum_{j=0}^{i} \sum_{k=1}^{\frac{n}{2^j}} 1 = \sum_{i=0}^{n-1} \sum_{j=0}^{i} \bigg( \dfrac{n}{2^j} \bigg) $$
From here we see the number of iterations that will be performed is triangular in a sense.  That is, the number of iterations is represented as the sum $1 + 2 + \hdots + n$, which is represented in the baby gauss formula $\dfrac{n(n+1)}{2}$, which is clearly of $O(n^2)$.  

\end{enumerate}


\item (10 points)
Describe an algorithm that produces the maximum, the minimum, and the mean, 
of a set of $n$ real numbers. Give an other algorithm that finds the median
of this set. Give their time complexities.

\textbf{Solution: }

This first algorithm will produce the max, min, and mean of a dataset in the form of an array.  Also, I'm assuming that the size of the set of numbers, $n$, is passed in as a parameter.  

\begin{verbatim}
Algorithm1 (input: Array numbers[0] ... numbers[n-1], length n)
max = numbers[0]
min = numbers[0]
sum = 0
for i = 0 to n-1
    if (numbers[i] >= max)
        max = numbers[i]
    if (numbers[i] <= min)
        min = numbers[i]
    sum = sum + numbers[i]
mean = sum / n
// Output three values
return min, max and mean
\end{verbatim}
This Algorithm linearly searches through the input of numbers, and keeps track of the biggest and smallest numbers it's seen so the time complexity is linear or $O(n)$.  

The second algorithm returns the median of a dataset.  This is done by sorting the set in ascending order and then depending on whether $n$ is even or odd, returning the middle element or the average of the two middle terms.  
\newpage
\begin{verbatim}
Algorithm2 (input: Array numbers[0] ... numbers[n-1], length n)
// Begin by sorting the array
for i = 0 to n-1
    for j = i + 1 to n-1
        if (numbers[j] < numbers[i])
            tmp = numbers[j]
            numbers[j] = numbers[i]
            numbers[i] = tmp
// At this point the array is sorted, we just need to return median
if (n % 2 == 1) // if n is odd
    median = numbers[roundup(n/2) - 1]
if (n % 2 == 0) // if n is even
    median = numbers[n/2 - 1] + numbers[n/2]
    median = median / 2
return median
\end{verbatim}
Because this algorithm is essentially a sorting algorithm with a couple calculations at the end, it will have the same time complexity as the sorting algorithm implemented.  Since we used bubble sort, and it's worst case complexity is $O(n^2)$, so is Algorithm2.  


\item (10 points)
Use bubble sort to sort 8, 5, 1, 4, 3, 2, showing the lists obtained 
at each outer step.

\textbf{Solution: }

Using bubble sort, the given list is sorted as follows:
$$ 8, 5, 1, 4, 3, 2 $$
$$ 5, 1, 4, 3, 2, 8 $$
$$ 1, 4, 3, 2, 5, 8 $$
$$ 1, 3, 2, 4, 5, 8 $$
$$ 1, 2, 3, 4, 5, 8 $$
$$ 1, 2, 3, 4, 5, 8 $$

\item (10 points)
Use insertion sort to sort 8, 5, 1, 4, 3, 2, showing the lists obtained 
at each outer step.

\textbf{Solution: }

Using insertion sort, the given list is sorted as follows:
$$ 8, 5, 1, 4, 3, 2 $$
$$ 5, 8, 1, 4, 3, 2 $$
$$ 1, 5, 8, 4, 3, 2 $$
$$ 1, 4, 5, 8, 3, 2 $$
$$ 1, 3, 4, 5, 8, 2 $$
$$ 1, 2, 3, 4, 5, 8 $$

\end{enumerate}

\end{document}

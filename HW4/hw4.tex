\documentclass[11pt]{article}

\usepackage{times,mathptm}
\usepackage{pifont}
\usepackage{exscale}
\usepackage{latexsym}
\usepackage{amsmath}
\usepackage{epsfig}
\usepackage{amsthm}
\usepackage{amssymb}

\textwidth 6.5in
\textheight 9in
\oddsidemargin -0.0in
\topmargin -0.0in

\parindent 0pt     % How much the first word of a paragraph is indented. 
\parskip 0pt       % How much extra space to leave between paragraphs.

\begin{document}

\begin{center}             % If you only centering 1 line use \centerline{}
\begin{LARGE}
{\bf CSci 243 Homework 4}
\end{LARGE}
\vskip 0.25cm      % vertical skip (0.25 cm)

Due: 10:00 am, Wednesday, Sep 30 \\  % force new line
Alexander Powell
\end{center}

\begin{enumerate}

\item (12 points)
For each of these partial sequences of integers, determine the next term of 
the sequence, and then provide a general formula or rule to generates terms of the
sequence.
\begin{enumerate}
\item $3, 6, 11, 18, 27, 38, 51, 66, 83, 102, \ldots$

The pattern that this sequence follows is $n^2 + 2$ where the first element has index $n = 1$.  Therefore, the next element in the sequence is $123$.  

\item $7, 12, 17, 22, 27, 32, 37, 42, 47, 52, 57, \ldots$

The pattern of this sequence is $2 + 5n$ where the first element has index $n = 1$.  Following this rule, the next term in the sequence will be $62$.  

\item $1, 2, 2, 2, 3, 3, 3, 3, 3, 5, 5, 5, 5, 5, 5, 5, \ldots$

This sequence follows the pattern of the fibonacci numbers in groups of $2n - 1$.  To clarify, the first number in the fibonacci sequence, $1$, will only appear one time because $2(1) - 1 = 1$.  This is then followed by three $2s$ because $2(2) - 1 = 3$ and next five $3s$ because $2(3) - 1 = 5$.  Following this pattern, the next term will be an $8$, but more specifically, there will be nine $8s$, because $2(5) - 1 = 9$.

\item $3, 9, 81, 6561, 43046721 \ldots$

The pattern of this sequence can be thought of as recursive because every term is the previous term squared, and the pattern starts at $3$.  Following the rule, the next term in the sequence will be $185302018881841$.  

\end{enumerate}

\item(4 points) Compute each of these double sums.
\begin{enumerate}
\item $\sum_{i=1}^2\sum_{j=2}^4(i+2j) = (1 + 2 \times 2) + (1 + 2 \times 3) + (1 + 2 \times 4) + (2 + 2 \times 2) + (2 + 2 \times 3) + (2 + 2 \times 4) = 45$


\item $\sum_{i=0}^2\sum_{j=0}^3(i+3j) = (0 + 3 \times 0) + (0 + 3 \times 1) + (0 + 3 \times 2) + (0 + 3 \times 3) + $

$(1 + 3 \times 0) + (1 + 3 \times 1) + (1 + 3 \times 2) + (1 + 3 \times 3) + $

$(2 + 3 \times 0) + (2 + 3 \times 1) + (2 + 3 \times 2) + (2 + 3 \times 3) = 66$


\item $\sum_{i=1}^3\sum_{j=0}^2 i = 1 + 1 + 1 + 2 + 2 + 2 + 3 + 3 + 3 = 18$


\item $\sum_{i=0}^2\sum_{j=1}^3 ij^2 = (0 \times 1^2) + (0 \times 2^2) + (0 \times 3^2) + $

$(1 \times 1^2) + (1 \times 2^2) + (1 \times 3^2) + $

$(2 \times 1^2) + (2 \times 2^2) + (2 \times 3^2) = 42$

\end{enumerate}

\newpage

\item(6 points) Compute each of these sums.
\begin{enumerate}
\item $\sum_{i=0}^n 5^{i+1}-5^i$

By taking out a constant $5$ and changing the exponent of the first sum, we can write the expression as follows:

$5 \times \sum_{i=0}^n 5^i - \sum_{i=0}^n 5^i$

Now, from the geometric series rule, we can rewrite the expression in terms of $n$:
$$ 5 \times \frac{5^{n+1}-1}{5-1} - \frac{5^{n+1}-1}{5-1} $$
$$ = 4 \times \frac{5^{n+1}-1}{4} = 5^{n+1} - 1$$

\item $\sum_{i=0}^{2n} (-3)^i$  (hint: split series in two parts)

By splitting the sum into two parts, we get the following:
$$ \sum_{i=0}^{2n} (-3)^i = \sum_{i=0}^{n} (-3)^i + \sum_{i = n+1}^{2n} (-3)^i $$
By using the geometric series rule, we can write the first sum as $\bigg( \frac{(-3)^{n+1} - 1}{(-4)} \bigg)$.  

Also, because we know that for cases where the geometric series does not begin at $i = 0$, we have the following formula:
$$ \sum_{i = m}^{\infty} r^k = \frac{r^m}{1-r} $$
By converting the second half of the previous sum into the given form we get:
$$ \Bigg( \bigg( \frac{(-3)^{n+1}}{4} \bigg) - \bigg( \frac{(-3)^{2n+1}}{4} \bigg) \Bigg) $$
After putting it all together, we get:
$$ \Bigg( \frac{(-3)^{n+1} - 1}{(-4)} \Bigg) + \Bigg( \bigg( \frac{(-3)^{n+1}}{4} \bigg) - \bigg( \frac{(-3)^{2n+1}}{4} \bigg) \Bigg) $$
With a little simplification we have
$$ \frac{(-3)^{n+1} - 1 - (-3)^{n+1} + (-3)^{2n+1}}{-4} = \frac{(-3)^{2n+1} - 1}{-4} $$

Alternatively, we could have not split the series into two parts and used the geometric series formula as follows:
$$ \frac{(-3)^{2n+1} - 1}{(-3) - 1} = \frac{(-3)^{2n+1} - 1}{-4} $$
Which is what we originally found.  

\end{enumerate}

\newpage

\item Consider the series $\sum_{k=2}^{2n+1}\frac{2}{k^2-1}.$
\begin{enumerate}
\item (4 points) Write the series as a telescoping series.

We can begin by rewriting the fraction $\frac{2}{k^2-1}$ as $\frac{2}{(k+1)(k-1)}$ which can be manipulated to be $\frac{k+1-k+1}{(k+1)(k-1)}$ or $\frac{(k+1)-(k-1)}{(k+1)(k-1)}$ which equals $\frac{(k+1)}{(k+1)(k-1)} - \frac{(k-1)}{(k+1)(k-1)}$.  This expression can then be transformed into a telescoping series defines as:
$$ \sum_{k=2}^{2n+1} \frac{1}{k-1} - \frac{1}{k+1} $$

\item (6 points) show 
$$\sum_{k=2}^{2n+1}\frac{2}{k^2-1}= \frac{3}{2}-\frac{1}{2n+1}-\frac{1}{2n+2}$$
Hint: write out at least the first six terms and the last two terms, and group them in pairs of two.

In the previous question we were able to transform the series to a telescoping series of the form $\frac{1}{k-1} - \frac{1}{k+1}$.  Using this form, we can write out the first terms as:
$$ \bigg(\frac{1}{1} - \frac{1}{3}\bigg) + \bigg(\frac{1}{2} - \frac{1}{4}\bigg) + \bigg(\frac{1}{3} - \frac{1}{5}\bigg) + \bigg(\frac{1}{4} - \frac{1}{6}\bigg) + \bigg(\frac{1}{5} - \frac{1}{7}\bigg) + \bigg(\frac{1}{6} - \frac{1}{8}\bigg) + \bigg(\frac{1}{7} - \frac{1}{9}\bigg)$$
$$ + \dots + \bigg(\frac{1}{2n-1} - \frac{1}{2n+1}\bigg) + \bigg(\frac{1}{2n} - \frac{1}{2n+2}\bigg)$$
From a little examination, we see that all of the terms cancel out except for the first, third, last, and third to last terms.  This leaves us with the following:
$$ 1 + \frac{1}{2} - \frac{1}{2n+1} - \frac{1}{2n+2} = \frac{3}{2} - \frac{1}{2n+1} - \frac{1}{2n+2} $$
and therefore we can conclude that 
$$\sum_{k=2}^{2n+1}\frac{2}{k^2-1}= \frac{3}{2}-\frac{1}{2n+1}-\frac{1}{2n+2}$$

\end{enumerate}

\newpage

\item (8 points) Prove by induction that $\sum_{i=1}^ni^2=\frac{1}{6}n(n+1)(2n+1)$.

\begin{proof}

Base case: when $i = 1$ we have that $1^2 = 1$ and also that $\frac{1}{6}(1)(1+1)(2(1)+1) = \frac{6}{6} = 1$, so the base case holds.  
Now, let's assume that this is true when $n=k$.  Then we have that
$$ \sum_{i=1}^{k} i^2 = \frac{1}{6}k(k+1)(2k+1) $$
When $n = k + 1$, we get the following:
$$ \sum_{i=1}^{k+1} i^2 = \sum_{i=1}^{k} i^2 + (k+1)^2 = \frac{1}{6}k(k+1)(2k+1) + (k+1)^2$$
$$ = \frac{1}{6}(k+1)(k(2k+1)+6(k+1)) = \frac{1}{6}(k+1)(2k^2+7k+6) $$
$$ = \frac{1}{6}(k+1)(2k+1)(k+2) = \frac{1}{6}(k+1)((k+1)+1)(2(k+1)+1), $$
Therefore, since the base case is valid and we have shown it is valid for $n=k+1$, we can conclude that it is valid for all $n \geq 1$.  

\end{proof}

\end{enumerate}

\end{document}






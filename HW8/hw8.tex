\documentclass[11pt]{article}

\usepackage{times,mathptm}
\usepackage{pifont}
\usepackage{exscale}
\usepackage{latexsym}
\usepackage{amsmath}
\usepackage{amssymb}
\usepackage{amsthm}
\usepackage{epsfig}

\textwidth 6.5in
\textheight 9in
\oddsidemargin -0.0in
\topmargin -0.0in

\parindent 0pt     % How much the first word of a paragraph is indented. 
\parskip 0pt       % How much extra space to leave between paragraphs.

\begin{document}

\begin{center}             % If you only centering 1 line use \centerline{}
\begin{LARGE}
{\bf CSci 243 Homework 8}
\end{LARGE}
\vskip 0.25cm      % vertical skip (0.25 cm)

Due: 10:00 am, Monday, November 2 \\  % force new line
Alexander Powell
\end{center}

\begin{enumerate}

\item (10 points)
Suppose that a store offers gift certificates in denominations of \$25 and
\$40. Determine the possible total amounts you can form using these gift
certificates. Prove your answer using strong induction.

\textbf{Solution: }
\begin{proof}

Let's begin by looking at possible combinations of gift certificates to see if we can spot a pattern.  This will be our inductive basis.  We have:
\begin{flushleft}
\small 25 $\times$ 1 = 25

\small 40 $\times$ 1 = 40

\small 25 $\times$ 2 = 50

\small 40 + 25 = 65

\small 25 $\times$ 3 = 75

\small 40 $\times$ 2 = 80

\small 40 + 25 $\times$ 2 = 90

\small 25 $\times$ 4 = 100

\small 40 $\times$ 2 + 25 = 105

\small 40 $\times$ 3 = 120

\small 25 $\times$ 5 = 125

\small 40 $\times$ 2 + 25 $\times$ 2 = 130

\small 25 $\times$ 4 + 40 = 140

\small 25 + 40 $\times$ 3 = 145

\small 25 $\times$ 6 = 150

\small 25 $\times$ 3 + 40 $\times$ 2 = 155

\small 40 $\times$ 4 = 160
\end{flushleft}

Now, at this point we can make the claim that we can create $140 + 5n, n \geq 0$ using the gift certificates.  
Inductive Hypothesis: We can form any number that's a multiple of $5$ and above $160$ with only $\$25$ and $\$40$ gift certificates.  
Inductive Step: We already have a string of $5$ consecutive multiples of $5$ ($140$ through $160$).  If $n$ is a multiple of $5$ greater than $160$ then the inductive hypothesis says that it is possible to create $n-25$ with the gift certificates.  Therefore, It is also possible to create $n$ by just adding another $\$25$ gift certificate.  Therefore, we have proven that we can form all amounts in the form $140 + 5n, n \geq 0$ as well as those shown in the base case.  
\end{proof}

\newpage
\item  (8 points)
Consider the dot game we saw in class. Again there are two rows
of dots, with $n_1$ and $n_2$ dots respectively, and players can remove
any number of dots during their turn, but only from one row.
However, now, the player who removes the last dot loses.

If $n_1=n_2 > 1$, prove that there is a winning strategy for the second player.
What happens if $n_1 \not = n_2$?

\textbf{Solution: }
\begin{proof} We use induction on $n$ (or equivalently $n_1$ and $n_2$).

\textbf{Base Case: } $n = 2$.  When $n_1 = n_2 = 2$ then there are $2$ rows of $2$ dots for a total of $4$ dots.  

There are two cases:

\textbf{Case 1: }

If player 1 removes one dot from $n_1$ then player 2 should remove both dots from $n_2$ forcing player 1 to remove the last dot on $n_1$
 and lose.  

\textbf{Case 2: }

If player 2 removes both dots from $n_1$ then player 2 should only remove one dot from $n_2$ forcing player 1 to remove the last dot on $n_2$.  Again, player 2 wins so the base step holds.  

\textbf{Inductive Hypothesis: } Assume true for any number of dots $j \leq k$, player 2 can choose a winning strategy.  

\textbf{Inductive Step: } Prove that for some games with rows of dots $k_1 = k_2 > 1$ and $k_1 = n_1 + 1$ there is a winning strategy 
for player 2.  From here there are three cases:

\textbf{Case 1: } If player 1 removes all dots except one from a row, then player 2 removes the entire other row, forcing player 1 to remove the last dot and lose.  

\textbf{Case 2: } If player 1 removes all dots from a row then player 2 should remove all but one dot from the other row, again forcing player 1 to remove the last dot and lose.  

\textbf{Case 3: } If player 1 leaves more than one dot on the row they play on, then player 2 should remove an equal number from the other row.  This will result in a smaller game that will eventually come to a point where we can use the inductive hypothesis to prove player 2 will have a winning strategy.  

Therefore, we have proven that if $n_1 = n_2 > 1$, there is a winning strategy for the second player.  

Also, if $n_1 \not = n_2$ there is not any guaranteed strategy for a player 2 to win.  Consider the game where there is a row of $3$ dots and a row of $2$ dots.  Then for the case where player 1 removes one dots from the row of three dots, either way the second player will lose (depending of course on the intelligence of the first player).  
\end{proof}
\item (6 points) Give a recursive definition for each of the following sequences
$\{a_n\}$ for $n=1, 2, 3, \ldots$.

\begin{enumerate}
\item $a_n=4n-2$

\textbf{Solution: }
We can see the pattern by evaluating the first few terms of the sequence:
$$ a_1 = 4(1) - 2 = 2 $$
$$ a_2 = 4(2) - 2 = 6 $$
$$ a_3 = 4(3) - 2 = 10 $$
$$ a_4 = 4(4) - 2 = 14 $$
So, we can see the the sequence $\{a_n\}$ is the sequence of starting at $2$, each successive element being $4$ more than the one before it.  Therefore, we can represent this sequence with the recursive definition:
$$ a_1 = 2,\text{ } a_{n+1} = a_n + 4 $$

\item $a_n=1+(-1)^n$

\textbf{Solution: }
We can see the pattern by evaluating the first few terms of the sequence:
$$ a_1 = 1 + (-1)^1 = 0 $$
$$ a_2 = 1 + (-1)^2 = 2 $$
$$ a_3 = 1 + (-1)^3 = 0 $$
$$ a_4 = 1 + (-1)^4 = 2 $$
So, we can see that the sequence is made up of alternating $0s$ and $2s$.  This can be represented with the recursive definition
$$ a_1 = 0,\text{ } a_{n+1} = a_n + 2(-1)^{n+1} $$

\item $a_n=(\frac{1}{2})^n$

\textbf{Solution: }
Again, by examining the pattern from the first few terms of the sequence we get:
$$ a_1 = \bigg(\frac{1}{2}\bigg)^1 = \frac{1}{2} $$
$$ a_2 = \bigg(\frac{1}{2}\bigg)^2 = \frac{1}{4} $$
$$ a_3 = \bigg(\frac{1}{2}\bigg)^3 = \frac{1}{8} $$
$$ a_4 = \bigg(\frac{1}{2}\bigg)^4 = \frac{1}{16} $$
So, the pattern is each element of the sequence is one half of the one before it.  This can be defined in the recursive definition:
$$ a_1 = \frac{1}{2},\text{ } a_{n+1} = \dfrac{a_n}{2} $$

\end{enumerate} 

\item For string $w=a_1a_2\cdots a_n$, the reversal of the string is defined as
$w^R=a_n\cdots a_2a_1$.
\begin{enumerate}
\item (2 points) What is $\epsilon^R$? What is $(10110)^R$?

\textbf{Solution: }
$\epsilon^R = \epsilon$ and $(10110)^R = (01101)$

\item (4 points) Give a recursive definition of the reversal of a string.

\textbf{Solution: }
A recursive definition for the reversal of a string can be given with the basis of $S^R = S$ and the recursive step: if $w = ua$ for $u \in \Sigma^*$ and $a \in \Sigma$, then $w^R = au^R$.  

\item (6 points) Use structural induction to prove that $(w_1w_2)^R=w_2^Rw_1^R$.

\textbf{Solution: }
Let $P(w)$ be the property where $(w_1w_2)^R=w_2^Rw_1^R$, when $w_1 \in \Sigma^*$.  
We must show that $\forall w \in \Sigma^*, \text{ } P(w)$.  

\textbf{Base case: } When $|w_2| = 0$, then $(w_1w_2)^R = w_1^R = w_2w_1^R = w_2^Rw_1^R$, so the base case holds.  

\textbf{Inductive Hypothesis: } Assume $P(w)$ is true for any $w$.  

\textbf{Inductive Step: } We must show that $P(wa)$ is true $\forall a \in \Sigma$.  From our assumption, $(w_1w_2)^R = w_2^Rw_1^R \text{ } \forall w \in \Sigma^*$.  So, $(w_1(w_2a))^R = ((w_1w_2)a)^R = a(w_1w_2)^R = aw_2^Rw_1^R = (w_2a)^Rw_1^R$.  Therefore, $P(wa)$ holds.  Thus, $(w_1w_2)^R = w_2^Rw_1^R$.  


\end{enumerate}

\item (8 points) 
A palindrome is a string that reads the same forward and backward, i.e., $w=w^R$.
Give a recursive algorithm in pseudocode that checks whether a given string $w$ is a palindrome.
What is the time complexity of your algorithm?

\textbf{Solution: }
\begin{verbatim}
Algorithm IsPalindrome(string word)
{
    // We consider 1 element strings to be palindromes
    if (length(word) <= 1) { return true }
    first = word[0]
    last = word[length(word) - 1]
    if (first != last) { return false }
    else
    {
        return IsPalindrome(word.substring(1, length(word) - 1))
    }
}
\end{verbatim}
If we denote $n$ to be the length of the input string then in the worst case the function would have to be recursively called $\frac{n}{2}$ times.  So we can say the order is $O(\frac{n}{2})$, which is equivalent to taking out the constant and saying the complexity is $O(n)$.  

\item (8 points) 
Give a recursive algorithm in pseudocode that finds the maximum number among $n$ integers.
What is the time complexity of your algorithm? 

\textbf{Solution: }
\begin{verbatim}
Algorithm GetMaximum(Array array, int n)
{
    if (n == 1) { return array[0] }
    else
    {
        max = GetMaximum(array, n-1)
        if (array[n-1] > max) { return array[n-1] }
        else { return max }
    }
}
\end{verbatim}
Since at the worst case this algorithm will have to be called as many times as the length of the given array, the time complexity can be described as $O(n)$.  

\item (8 points) 
Assume $n=4^k$ (i.e., $k=\log_4 n$) for some $k$. 
Solve the following recurrence relation by iteration.

$$f(n)=\left\{\begin{array}{ll}
1&\mbox{if $n=1$}\\
3f({n\over4})+n&\mbox{if $n\ge2$}
\end{array}
\right.$$

\textbf{Solution: }
By evaluating the first few iterations of the recurrence relation we get:
$$ f(1) = 1 $$
$$ f(4) = 3 \times f(1) + 4 = 7 $$
$$ f(16) = 3 \times f(4) + 16 = 37 $$
$$ f(64) = 3 \times f(16) + 64 = 175 $$
$$ f(256) = 3 \times f(64) + 256 = 781 $$
$$ f(1024) = 3 \times f(256) + 1024 = 3367 $$

From this, we can hypothesize the recurrence relation can be represented using the following formula:
$$ 4^{log_4 (n \times 4)} - 3^{log_4 (n \times 4)} = 4^{1 + \log_4 n} - 3^{1 + \log_4 n} $$
We need to prove this with induction:

\textbf{Base Case: } $f(1) = 1 = 4^{1 + log_4 1} - 3^{1 + log_4 1} = 4^1 - 3^1 = 1$, so the base case holds true.  

\textbf{Inductive Hypothesis: } Assume for all $k$ that $f(k) = 4^{1 + \log_4 k} - 3^{1 + \log_4 k}$.  

\textbf{Inductive Step: } We must prove that for $4k$ the property holds true.  
We can show that $4^{1 + log_4 4k} - 3^{1 + log_4 4k} = 3 \times f(\frac{4k}{k}) + 4k$.  Additionally, 
$$ f(4k) = 4 f(\frac{4k}{k}) + 4k = 3 (4 ^{1 + \log_4 k} - 3^{1 + \log_4 k}) + k = 3 \times 4^{1 + \log_4 k} - 3^{2 + \log_4 k} + 4k $$
$$ = 3 \times 4^{1 + \log_4 k} - 3^{1 + \log_4 4k} + 4k = 4k + 4k - 3^{1 + \log_4 4k} $$
$$ = 12k + 4k - 3^{1 + \log_4 4k} = 16k - 3^{1 + \log_4 4k} = 4^{1 + \log_4 4k} - 3^{1 + \log_4 4k}, $$

Therefore, we have proven that the recurrence relation can be solved with the above closed formula.  

\end{enumerate}
\end{document}





